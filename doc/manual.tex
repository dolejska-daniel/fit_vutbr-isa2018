\documentclass[11pt]{article}
\usepackage[utf8]{inputenc}
\usepackage[T1]{fontenc}
\usepackage[main=czech]{babel}
\usepackage[nottoc]{tocbibind}
\usepackage{url}
\usepackage{xcolor}
\usepackage{minted}
\usepackage{array}

\usepackage{parskip}
\setlength{\parindent}{20pt}

\title{\textbf{FIT VUT Brno - ISA}\\
	Export DNS informací pomocí protokolu Syslog}
\author{Daniel Dolejška, \texttt{xdolej08@stud.fit.vutbr.cz}}
\date{\today}

\begin{document}
	
	\maketitle
	\tableofcontents
	
	
	\newpage
	
	\section{Úvod}
	Krátký všeobecný úvod do problematiky služby DNS a Syslog.
	
	\subsection{DNS}
	Služba DNS patří mezi kritické služby dnešního internetu - slouží především, ale ne pouze, pro překlad doménových názvů na IP adresy a naopak. Adresy IPv4 jsou uloženy v DNS záznamech typu \texttt{A}, IPv6 v záznamech typu \texttt{AAAA}\cite{RFC1035}.
	
	Dále se používá pro identifikaci existujících služeb v rámci daných domén, například odesílání emailů (Mail Exchange - \texttt{MX}), či jiných\cite{RFC2782} (\texttt{SRV}).
	
	Pro zajištění vyšší bezpečnosti a ochrany před podvržením těchto záznamů vzniklo pro službu DNS rozšíření s názvem DNSSEC, které pracuje na principu podepisování záznamů\cite{RFC4034}.
	
	\subsection{Syslog}
	Syslog je standard pro logovací zprávy\cite{RFC5424}. Dovoluje filtrování na základě zdroje i závažnosti zpráv.
	
	
	\section{Implementace}
	Popis řešení problémů a jejich konkrétní implementace v rámci této aplikace.
	
	Celé řešení tohoto projektu je postaveno výhradně na vlastních zdrojových textech a nevyužívá žádné knihovny třetích stran. Je zpracováno v programovacím jazyce C, drží se standardu C99.
	
	\subsection{Přijímání dat}
	\subsubsection{Data ze síťového rozhraní}
	Pro přístup k síťové komunikaci na daných síťových rozhraních jsou použity tzv. \textsf{raw sockety}.
	
	Tento typ schránky nám dává přístup k veškerému síťovému provozu pro daná síťová rozhraní na velmi nízké úrovni - systém neprovádí žádné úpravy dat ani jejich zpracování a dává nám přístup k datovému byte streamu\cite{Linux-raw}.
	
	Tato data jsou dále podstoupena ke zpracování, informace ke zpracovávání síťového provozu jsou v kapitole~\ref{section:data-processing}.
	
	\subsubsection{Soubor \textsf{pcap} (\textsf{tcpdump})}
	Zpracování souboru \textsf{.pcap} probíhá díky jeho struktuře velice jednoduše a dovoluje téměř stejné zacházení jako přímé čtení komunikace na síťovém rozhraní - jedná se o čistý byte stream předdefinovaných datových struktur a samotných dat. Oproti síťovému provozu se ale liší\cite{WiresharkWiki-Libpcap}:
	
	\begin{itemize}
		\item \textbf{Globální hlavička} \\
		Soubor obsahuje záhlaví, ve kterém se nacházejí méně důležité informace typu: verze formátu souboru, údaje o časové zóně aj.
		
		\item \textbf{Záhlaví packetu} \\
		Každá zachycená síťová komunikace pak má vlastní hlavičku, která obsahuje mnohem důležitější informace - například čas zachycení či délku vlastní komunikace (packetu).
	\end{itemize}
	
	{
		\setlength\extrarowheight{2pt}
		\begin{table}[H]
			\makebox[\textwidth][c]{
				\begin{tabular}{|c|c|c|c|c|c|}
					\hline
					\texttt{Globální hlavička} & \texttt{Záhlaví packetu} & \texttt{Data packetu} & \texttt{Záhlaví packetu} & \texttt{Data packetu} & \texttt{\ldots} \\ \hline
				\end{tabular}
			}
			\caption{Formát \textsf{pcap} souboru\cite{WiresharkWiki-Libpcap}} \label{table:pcap-file-format}
		\end{table}
	}

	Veškerá zachycená komunikace je v souboru uložena bezprostředně za sebou umožňující její snadné zpracování, viz Tabulka~\ref{table:pcap-file-format}. Program dále předpokládá validitu souboru -- nejsou prováděny žádné kontroly a validace.
	
	Zpracování probíhá v několika krocích - otevření souboru, přečtení globální hlavičky, opakované čtení záhlaví packetu spolu s jeho daty (délka je uvedena v záhlaví) dokud není dosažen konec souboru.
	
	Získaná data síťového provozu jsou následně předána ke zpracování, kde je s nimi zacházeno naprosto stejným způsobem jako s daty ze síťového rozhraní - více v kapitole~\ref{section:data-processing}.

	\subsection{Zpracování dat} \label{section:data-processing}
	\subsubsection{Packet} \label{section:data-processing:packet}
	Pro účely zpracování dat z raw socketů jsem použil a rozšířil základ z loňského projektu do předmětu IPK. Zpracování využívá vestavěné struktury popisující Ethernet, IP\cite{RFC0791}, UPD\cite{RFC0768} i TCP\cite{RFC0793} hlavičky dle definovaných standardů.
	
	Z IP hlavičky je nejdříve zjištěno, zda se jedná o packet příslušící protokolu DNS - byl odeslán z/na port \textsf{53}\cite{RFC1035}. Přenosy neobsahující odpovědi DNS serveru jsou aplikací ignorovány. Z IP hlavičky je dále zjištěno jaký transportní protokol (UDP či TCP) byl pro přenos použit -- na základě této informace se zpracování packetu dále dělí:
	
	V případě, že byl pro přenos použit protokol TCP aplikace před dalším zpracováním čeká na chybějící data, pokud je to nutné\cite{RFC7766}. Odpovědi jsou párovány na základě sekvenčního čísla packetu\cite{RFC0793}. Po sestavení celé DNS odpovědi pokračuje aplikace ve zpracovávání těchto dat (Kapitola~\ref{section:data-processing:dns}).
	
	Pro DNS odpovědi přenášené protokolem UDP je dalším krokem jejich okamžité zpracování (Kapitola~\ref{section:data-processing:dns}).
	
	\subsubsection{DNS} \label{section:data-processing:dns}
	Ze získaných dat jsou nejdříve zpracovány statické položky: ID transakce, počty obsažených odpovědí a další doplňující informace\cite{RFC1035}.
	
	Na základě počtů, získaných z DNS hlavičky, jsou následně zpracovávány jednotlivé DNS záznamy dle jejich specifikací\cite{RFC1035}\cite{RFC4034}. Položky složitějších záznamů jsou v textové podobě reprezentovány číselně oddělené mezerami.
	
	\subsection{Vyhodnocování dat}
	Kompletně zpracované odpovědi DNS serverů jsou po jedné vkládány do tabulky s rozptýlenými položkami. Jako vyhledávací klíč je použita textová reprezentace dané odpovědi - typicky v následujícím formátu:
	
	\begin{minted}{xml}
<domain-name> <rr-type> <rr-data>
	\end{minted}
	
	Jako hodnota je pro daný klíč použito číslo reprezentující počet těchto odpovědí. Při prvním vložení DNS odpovědi je v tabulce inicializována s hodnotou 1. Při každém dalším vložení je její hodnota inkrementována o 1.
	
	Do statistik jsou zahrnuty odpovědi uvedené v answers či authority (additionals jsou aplikací ignorovány) avšak pouze následující podporované typy:
	
	\makebox[\textwidth][c]{
		\texttt{A}, \texttt{AAAA}, \texttt{NS}, \texttt{PTR}, \texttt{CNAME}, \texttt{SRV}, \texttt{KX}, \texttt{MX}, \texttt{TA}, \texttt{DLV}, \texttt{DS}, \texttt{SOA}, \texttt{NSEC}, \texttt{NSEC3}, \texttt{RRSIG}, \texttt{DNSKEY}, \texttt{KEY}, \texttt{TXT}
	}
	
	Po každém odeslání/výpisu statistik (na \texttt{stdout}/syslog server)) z tabulky je tabulka vyčištěna a výpočet probíhá nanovo.
	
	V projektu jsem použil svou implementaci tabulky s rozptýlenými položkami, kterou jsem naprogramoval v rámci domácí úlohy do předmětu IAL.
	
	
	\subsection{Odesílání dat} \label{section:data-sending}
	\subsubsection{Výpis - \texttt{stdout}}
	Po uplynutí daného časového intervalu (pokud není nastaveno odesílání statistik na syslog server) nebo po přijetí signálu \texttt{USR1} dojde k výpisu aktuálních statistik na standardní výstup (\texttt{stdout}).
	
	Statistiky jsou vypisovány v následujícím formátu (čísla v hranatých závorkách udávají počet výskytů v rámci jednoho výpisu):
	\begin{minted}{xml}
=== DNS Traffic Statistics (last %ld minute(s) %ld second(s)) ===\n [1]
<domain-name> <rr-type> <rr-data> <count>\n [0-n]
\n [1]
	\end{minted}
	
	Pokud k výpisu došlo kvůli vypršení časového intervalu, je tabulka se statistikami vyprázdněna, jinak k vyprázdnění nedochází.
	
	\subsubsection{Syslog server}
	Po uplynutí daného časového intervalu (pokud je nastaveno odesílání statistik na syslog server) nebo po dokončení zpracování souboru (pokud není explicitně nastaven jiný časový interval než výchozí) dojde k odeslání aktuálních statistik na specifikovaný syslog server.
	
	Zprávy nejsou na syslog server odesílány samostatně - aplikace se snaží poskládat co nejvíce po sobě jdoucích zpráv do jednoho přenosu v maximální délce 1024 znaků. Seskupené zprávy jsou mezi sebou v rámci daného packetu odděleny pomocí CRLF (\texttt{0x0d 0x0a}). Jednotlivé zprávy se řídí následujícím formátem:
	\begin{minted}{xml}
<134> 1 YYYY-mm-ddTHH:ii:ss.000Z <hostname> dns-export <pid> - - <message>
	\end{minted}
	
	Zpráva je odeslána protokolem UDP na port 514\cite{RFC5424}. Po odeslání je tabulka se statistikami vyprázdněna.
	
	
	\section{Práce s programem}
	Přehled toho, co lze s aplikací dělat je k dispozici v souboru \textsf{README.md}.
	
	\newpage
	\bibliographystyle{plain}
	\bibliography{zdroje}
	
\end{document}
